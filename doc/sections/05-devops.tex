\section{Devops}
\frame{\tableofcontents[currentsection]}

%-------------------------------------------------
\subsection{DVCS Strategy}
\begin{frame}
    \frametitle{DVCS Strategy}
    Per gestire i repository sono state effettuate le seguenti scelte:
    \begin{itemize}
        \item Definizione di una \textbf{GitHub Organization} per contenere tutti i progetti.
        \item Utilizzo di \textbf{Git Flow Workflow} per avere una linea di sviluppo chiara e ben definita.
        \item Utilizzo dei \textbf{Conventional Commit} per poter generare in modo automatico i nuovi \textbf{tag} e le relative \textbf{release}
        \item Configurazione di un \textbf{Commit Linter} per assicurare che i commit siano correttamente formattati
    \end{itemize}
\end{frame}
%------------------------------------------------

%-------------------------------------------------
\subsection{Continuous Integration}
\begin{frame}
    \frametitle{Continuous Integration}
    All'interno dei repository dell'organizzazione sono stati configurati dei workflow che, tramite delle GitHub Actions,
    effettuano i seguenti controlli:
    \begin{itemize}
        \item \textbf{Code Quality Control}: tramite i plugin \texttt{ktlint} e \texttt{ESLint}.
        \item \textbf{Testing}: utilizzando i plugin \texttt{kotest} e \texttt{mocha}.
        \item \textbf{Reporting della coverage}: con i plugin \texttt{jacoco} e \texttt{istanbul}.
    \end{itemize}

    \bigskip

    Inoltre è stato configurato il bot \texttt{renovate} all'interno dei repository per effettuare \textbf{Automatic Dependency Update}.

\end{frame}

%------------------------------------------------

%-------------------------------------------------
\subsection{Continuous Delivery}
\begin{frame}
    \frametitle{Continuous Delivery}
    Sono stati inoltre configurati dei workflow che consentono di effettuare:
    \begin{itemize}
        \item \textbf{Semantic Versioning and Releasing}: GitHub Action che consente di calcolare automaticamente il tag della release ed effettuarla tramite l'analisi dei messaggi di commit.
        \item \textbf{Containerization}: L'applicazione viene costruita utilizzando un Dockerfile e pubblicata all'interno dei GitHub Packages.
    \end{itemize}
\end{frame}


%------------------------------------------------

