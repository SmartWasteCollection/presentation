\section{Requisiti}
\frame{\tableofcontents[currentsection]}

%-------------------------------------------------
\begin{frame}
    \frametitle{Requisiti}
    Per definire i requisiti del sistema che verrà sviluppato, sono state utilizzate due tecniche:

    \begin{itemize}
        \item \textbf{User Stories}: Per delineare quali sono i requisiti richiesti dal punto di vista degli utilizzatori
        del sistema.
        \item \textbf{Use Cases}: Per una rappresentazione tramite \textit{diagrammi dei casi d'uso} per evidenziare le
        interazioni tra i vari attori del sistema.
    \end{itemize}

\end{frame}
%------------------------------------------------

%-------------------------------------------------
\subsection{User Stories}

\begin{frame}
    \frametitle{User Stories}
    Le \textbf{User Stories} sono state definite dal punto di vista di \textbf{Managers}, \textbf{Citizens} e
    \textbf{Truck Drivers}, seguendo il pattern:

    \bigskip

    \begin{block}{}
        \begin{center}
            \texttt{\textbf{As a} ... \textbf{I want to} ... \textbf{So that} ...}
        \end{center}
    \end{block}

    \bigskip

    Ad esempio:

    \medskip

    \begin{addmargin}[1em]{2em}
        \textit{\textbf{As a} Citizen \textbf{I want to} book an "at home" waste collection \textbf{so that} I don't have to go to the disposal point. }
    \end{addmargin}

\end{frame}
%------------------------------------------------

%-------------------------------------------------
\subsection{Use Cases}

\begin{frame}
    \frametitle{Use Cases}
    Dopo aver analizzato le funzionalità dal punto di vista degli attori del sistema, sono stati definiti dei
    diagrammi UML che ne descrivono i comportamenti.

    \bigskip

    In particolare sono stati evidenziati le seguenti "macro-funzionalità":
    \begin{itemize}
        \item Gestione della raccolta dei \textbf{rifiuti ordinari}
        \item Gestione della raccolta dei \textbf{rifiuti straordinari}
        \item \textbf{Dashboard}
        \item Gestione dei \textbf{reclami}
    \end{itemize}
\end{frame}
%------------------------------------------------
